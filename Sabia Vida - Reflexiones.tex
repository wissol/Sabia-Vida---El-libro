\documentclass[12pt,spanish,a4paper,DIV=7,twoside=false]{scrbook}
% fuente base 12, en Español, din A4, la medida del área de texto que prefiero, a una sola cara,

% ------ Línea de 80 caracteres -----------------------------------------------

% Este documento es un Sabia Vida de libro en komascript, preparado para
% trabajar en español

% ------ lista de tareas ------------

\usepackage{todonotes}


% ------ ilustraciones --------------

\usepackage{graphicx}
\DeclareGraphicsRule{.tif}{png}{.png}{`convert #1 `dirname #1`/`basename #1 .tif`.png}


% ------ tipografía ---------------------

\usepackage{Alegreya} % U otra fuente que se desee
\addtokomafont{disposition}{\rmfamily} % Encabezados en Serif

\setkomafont{caption}{\footnotesize\itshape} % -- Etiquetes fuente pequeña
\setkomafont{captionlabel}{\usekomafont{caption}}

\usepackage{microtype}  % -- Arreglos Tipográficos

\usepackage[autostyle,spanish=spanish]{csquotes} %-- citas

\usepackage{verse} %-- poemas

\usepackage[toc]{multitoc} %-- índice general en 2 columnas

% ------ cuadros ---------------------

\usepackage{booktabs}
\usepackage{tabularx}
\usepackage{wrapfig}

% ------ idioma -----------------------

\usepackage[spanish,es-noindentfirst]{babel}  % en Español, por favor
% primer párrafo de cada sección sin sangría
\usepackage[utf8]{inputenc} % codificación de caracteres 
\usepackage[T1]{fontenc}


% ------- referencias -----------------

\usepackage[nottoc]{tocbibind}
\usepackage{cleveref} 
%  Documentación http://tug.ctan.org/macros/latex2e/contrib/cleveref/cleveref.pdf


% ------- enlaces ----------------------

\usepackage[	 colorlinks=true, pdfstartview=FitV, linkcolor=blue, 
                 	citecolor=blue, urlcolor=blue, pdfborder={0 0 0}, 
			   	pdftitle=Sabia Vida, pdfsubject=reflexiones,
				pdfauthor=Miguel de Luis, pdflang=es ES]{hyperref} %enlaces debe ir al final


% ------ meta -------------------------

\title{Sabia Vida}
\subtitle{Reflexiones}
\author{Miguel de Luis}
\date{\today} % o la fecha que sea o \today
\publishers{Pequeña Editorial Invisible}

\begin{document}


% ------ preliminares -----------------

\frontmatter
\maketitle       % --- Inserta página de título
\listoftodos     % --- tareas pendientes 
\todo{borrar lista de todos}

\tableofcontents % --- Inserta índice general


\chapter{Prefacio}

Mi intención es que éste sea el libro más importante de mi vida. No es un intento de sacar dinero de mi blog, ni tampoco recopilar los artículos de más éxito en forma de libro sino una reflexión más profunda y extensa sobre los temas que dieron origen a \textsc{Sabia Vida}

El primer peligro al que me enfrento es caer en algo tan pretencioso como construir una suerte de práxis filosófica para la que carezco de suficientes conocimientos. El segundo, --y al que temo más--, es quedarme corto, conformándome con los lugares comunes y las ideas abreviadas que repiten lo que ya se ha dicho y, en realidad, no necesita volver a decirse. Entre esos dos extremos pretendo situarme, con la esperanza de llegar al final algo de valor que sea digno de transmitirse.

% 140 palabras de 1500

\todo{terminar prefacio}

% ------ contenido principal ----------

\mainmatter

%\lorem[3]

\section{Ligula venenatis}

\lorem[6]

\begin{figure}
\begin{enumerate}
\item Aliquam convallis fringilla libero
\item et vestibulum eros viverra vitae
\item Donec semper hendrerit sapien in tempus
\item sit amet placerat nulla condimentum
\item Donec facilisis tempor arcu
\item Sed consequat finibus gravida
\end{enumerate}
\caption{Lista Numerada}
\end{figure}

\lorem[3]

\chapter{Duis mi velit}

\section{Cras cursus ante tellus}
\lorem[3]

\section{Nam vel finibus}

\lorem[8]


\chapter{¿Para qué vivir?}

Suena a pregunta de tristeza, pero que debería responderse, o al menos plantearse en tiempos de serenidad, y también de alegría, --sobre todo, de alegría. Sin embargo, casi nunca profundizamos en la pregunta, porque la respuesta se piensa, o mejor dicho, se \emph{siente} obvia en la alegría y la tristeza, e insondable en la serenidad. Mientras tanto, la vida sigue fluyendo, con, sin o contra nuestra consciencia. Es como si estuviéramos en una fiesta sin saber para qué estamos, como sin sufriéramos sin saber dónde está el dolor, o cómo si saltáramos de pensamiento en pensamiento.

Estoy convencido de que, --salvo en términos muy genéricos--, debo abstenerme de \emph{dar-te la} respuesta a esta cuestión. En otras palabras no voy a intentar pensar por ti, porque no aspiro a imposibles. Me conformaré con compartir mis meditaciones. %-- La fuente es la brújula
\chapter{Lo esencial es suficiente}

 %-- Quedándonos con lo esencial
\chapter{Alegre defensa de la levedad}

\begin{figure}
\begin{description}
\item{Ligereza}\\
De ligero.\\
1. f. Presteza, agilidad.\\
2. f. Levedad o poco peso de algo.\\
3. f. Inconstancia, volubilidad, inestabilidad.\\
4. f. Hecho o dicho de alguna importancia, pero irreflexivo o poco meditado.\\
\end{description}
\end{figure}

La definición de \emph{levedad} que tanto la Real Academia de la Lengua, como el Diccionario de uso del Español se remiten a la de \emph{ligereza} y ésta última parece tener mal prensa. La ligereza, y con ella la levedad, se nos presentan como cosa de tontos, de niños, o de apresurados. Parece ser precisamente lo contrario de lo defiendo tanto en mi blog como en mi libro. Es más, a veces pasa, que uno se lleva una idea equivocada de las palabras, a lo mejor se queda uno con el uso creativo que le ha dado un escritor, o escuchado en una película y acaba por adoptar el significado erróneo por el real. No estoy totalmente seguro de que eso no me haya pasado tmabién en este caso, pero aún a riesgo de empecinarme, ni voy a buscar otro término, ni inventarme otro, ni proponer otro significado. Deja, en vez de eso, que me explique.

Vivimos en una gigantesca nave espacial, un mundo que, a velocidades increíbles viaja, con todo el resto de la galaxia, en rumbo de encuentro, --que no de colisión, no en el sentido usual--, con Andrómeda. Al mismo tiempo damos vueltas junto con todo el resto del sistema en torno al centro de la Vía Lácteca, --con toda probabilidad un agujero negro--, y, casi por último, en torno al Sol, una estrella que, pasados unos miles de millones de años, se hinchará hasta \enquote{morir}. Las otras estrellas, --a las que para entonces podremos haber emigrado-- acabarán, tras procesos distintos, un destino en el que dejarán de producir luz y calor. 

El mismo universo es leve. No solo por lo inconstante de su futuro sino por que, en densidad, es efectivamente leve. Casi todo el universo es espacio ocupado, en todo caso, por espuma cuántica, algún átomo de hidrógeno despistado o, para la mayoría de los efectos, vacío. Además la materia que conocemos, con la honrosa excepción de las estrellas de neutrones, consiste en poco más que vacío ligeramente contaminado de partículas.

Sin irnos tan lejos en el tiempo, ni tan alejados de nosotros mismos, ¿qué hay más leve que el propio ser humano? 

\section{La alegre humildad}

La levedad que propongo es alegre humildad, y consiste en la aceptación de la verdad de la condición humana, frágil, pecadora, siniestra a veces, siempre inconstante, con aspiraciones inmensamente mayores que sus capacidades y, sin embargo, capaz de sorprendernos con las más grandes hazañas cuando solo cabía esperar mediocridades. Y esto es predicable tanto de la humanidad en su conjunto, como de cada especimen de nuestra especie. Quizás, la condición humana no sea otra cosa, que el reflejo de la condición del universo.

Ante esta verdad hay tres opciones: la soberbia, que niega los límites o se exige superarlos, a pesar de ser imposibles, la humildad resignada, que acepta los límites pero evitar pensar en ellos o sentirlos con cualquier sentimiento que no sea el de tristeza, y, la alegre humildad.

\begin{figure}
\settowidth{\versewidth}{los niños en el templo}
\begin{verse}[\versewidth]
descalzos viven\\
los niños en el templo\\
\vin mar y arena
\end{verse}
\end{figure}

% Somos cosa leve, cosa pequeña, cosa humilde, como los niños. ¡Juguemos! %-- Alegre defensa de la levedad
\chapter{La muerte}

\begin{figure}
\settowidth{\versewidth}{sus sombras las persiguen }
\begin{verse}[\versewidth]
\vin precipicio\\
el mar arranca sus rocas\\
\vin fin de la isla
\end{verse}
\end{figure}

Hablaremos de la muerte, como deberemos hablar del nacimiento. Si es que tiene algún sentido hablar de la vida, lo tendrá también hablar de sus dos límites, hablar del sentido de la vida es hablar del sentido de la muerte y viceversa.

Una propuesta particularmente popular es que la muerte carece de sentido; es sencillamente una desgracia que pone fin a una situación querida. Desde ese punto de vista la única reflexión que puede hacerse de la muerte es aprovechar la vida para asegurarse que se ha vivido antes de morir.

\section{Texto de Lorem}

\lipsum %-- Sobre la muerte

\backmatter

\appendix

\chapter{Resumen}

\end{document}
