\documentclass[12pt,spanish,a4paper,DIV=7,twoside=false]{scrbook}
% fuente base 12, en Español, din A4, la medida del área de texto que prefiero, a una sola cara,

% ------ Línea de 80 caracteres -----------------------------------------------

% Este documento es un Sabia Vida de libro en komascript, preparado para
% trabajar en español

% ------ lista de tareas ------------

\usepackage{todonotes}


% ------ ilustraciones --------------

\usepackage{graphicx}
\DeclareGraphicsRule{.tif}{png}{.png}{`convert #1 `dirname #1`/`basename #1 .tif`.png}


% ------ fuente ---------------------
\usepackage{Alegreya} % U otra fuente que se desee
\addtokomafont{disposition}{\rmfamily} % Encabezados en Serif

\setkomafont{caption}{\footnotesize\itshape} % -- Etiquetes fuente pequeña
\setkomafont{captionlabel}{\usekomafont{caption}}

\usepackage{microtype}  % -- Arreglos Tipográficos

\usepackage[autostyle,spanish=spanish]{csquotes} %-- citas

\usepackage{verse} %-- poemas

% ------ cuadros ---------------------

\usepackage{booktabs}
\usepackage{tabularx}
\usepackage{wrapfig}

% ------ idioma -----------------------

\usepackage[spanish,es-noindentfirst]{babel}  % en Español, por favor
% primer párrafo de cada sección sin sangría
\usepackage[utf8]{inputenc} % codificación de caracteres 
\usepackage[T1]{fontenc}


% ------- referencias -----------------

\usepackage[nottoc]{tocbibind}
\usepackage{cleveref} 
%  Documentación http://tug.ctan.org/macros/latex2e/contrib/cleveref/cleveref.pdf


% ------- enlaces ----------------------

\usepackage[	 colorlinks=true, pdfstartview=FitV, linkcolor=blue, 
                 	citecolor=blue, urlcolor=blue, pdfborder={0 0 0}, 
			   	pdftitle=Sabia Vida, pdfsubject=reflexiones,
				pdfauthor=Miguel de Luis, pdflang=es ES]{hyperref} %enlaces debe ir al final


% ------ meta -------------------------

\title{Sabia Vida}
\subtitle{Reflexiones}
\author{Miguel de Luis}
\date{\today} % o la fecha que sea o \today
\publishers{Pequeña Editorial Invisible}

\begin{document}


% ------ preliminares -----------------

\frontmatter
\maketitle       % --- Inserta página de título
\listoftodos     % --- tareas pendientes 
\todo{borrar lista de todos}
\tableofcontents % --- Inserta índice general

\chapter{Prefacio}

Mi intención es que éste sea el libro más importante de mi vida. No es un intento de sacar dinero de mi blog, ni tampoco recopilar los artículos de más éxito en forma de libro sino una reflexión más profunda y extensa sobre los temas que dieron origen a \textsc{Sabia Vida}

El primer peligro al que me enfrento es caer en algo tan pretencioso como construir una suerte de práxis filosófica para la que carezco de suficientes conocimientos. El segundo, --y al que temo más--, es quedarme corto, conformándome con los lugares comunes y las ideas abreviadas que repiten lo que ya se ha dicho y, en realidad, no necesita volver a decirse. Entre esos dos extremos pretendo situarme, con la esperanza de llegar al final algo de valor que sea digno de transmitirse.

% 140 palabras de 1500

\todo{terminar prefacio}

% ------ contenido principal ----------

\mainmatter

%\lorem[3]

\section{Ligula venenatis}

\lorem[6]

\begin{figure}
\begin{enumerate}
\item Aliquam convallis fringilla libero
\item et vestibulum eros viverra vitae
\item Donec semper hendrerit sapien in tempus
\item sit amet placerat nulla condimentum
\item Donec facilisis tempor arcu
\item Sed consequat finibus gravida
\end{enumerate}
\caption{Lista Numerada}
\end{figure}

\lorem[3]

\chapter{Duis mi velit}

\section{Cras cursus ante tellus}
\lorem[3]

\section{Nam vel finibus}

\lorem[8]


\chapter{¿Para qué vivir?}

\begin{figure}
\settowidth{\versewidth}{sus sombras las persiguen }
\begin{verse}[\versewidth]
\vin las nubes pasan \\
sus sombras las persiguen \\
\vin el cielo queda
\end{verse}
\end{figure}

Suena a pregunta de tristeza, pero que debería responderse, o al menos plantearse en tiempos de serenidad, y también de alegría, --sobre todo, de alegría. Sin embargo, casi nunca profundizamos en la pregunta, porque la respuesta se piensa, o mejor dicho, se \emph{siente} obvia en la alegría y la tristeza, e insondable en la serenidad. Mientras tanto, la vida sigue fluyendo, con, sin o contra nuestra consciencia. Es como si estuviéramos en una fiesta sin saber para qué estamos, como sin sufriéramos sin saber dónde está el dolor, o cómo si saltáramos de pensamiento en pensamiento.

Estoy convencido de que, --salvo en términos muy genéricos--, debo abstenerme de \emph{dar-te la} respuesta a esta cuestión. En otras palabras no voy a intentar pensar por ti, porque no aspiro ni a imposibles ni a estupideces. Me conformaré con compartir mis meditaciones, mostrarte a dónde yo he llegado, lo que he meditado, no para imponer mi visión y mi reflexión, pues no es perfecta ni para mí mismo sino para animarte a caminar también en este viaje.

\section{La fuente de la vida}

\begin{figure}
\blockquote[Sheldon Cooper]{
Trilema de Münchhaüsen: o la razón depende de una serie de subrazones, lo que lleva a un regresión infinita o se basa en determinados axiomas arbitrarios o forma un ciclo en sí misma, por ejemplo, me quiero ir porque me quiero ir.}
\end{figure}

La explicación del insigne y ficticio Dr Sheldon Cooper sobre el trilema de Münchhaüsen me sirve para facilitar la idea de que carecemos de un método científico para determinar los axiomas de la ciencia. En otras palabras debemos aceptar una base que nos parezca coherente, empezar a construir sobre ella y, si todo falla, volver a empezar. Si esto fuera la entrada de un \emph{blog} lo dejaría aquí, o me bastaría insistir un poco en la idea, añadiendo alguna vía para explorar esa fuente, quizás con la añadidura de la palabra Dios, pero aquí voy a ser mucho más osado. Lo habréis visto venir, la fuente de la vida, o, si queréis, la fuente de la vida que he encontrado, es Dios.

Dios y, en particular, el Dios cristiano, no es tendencia en occidente, pero debo mantenerme firme. Si fuera budista hablaría de el Buda y de sus enseñanzas y sería tu elección aceptarlas o rechazarlas; en todo o en parte. Pero soy cristiano y como tal debo hablar, sin pedir perdón por ello.

Encontré mi fuente no sin esfuerzo; no sé si tú pretendes que la tuya te venga caída del cielo, mientras estás dormido o distraído en otras cosas.

\section{La brújula}

Una vez que encuentres la fuente tendrás, como regalo, una brújula

% Una brújula sirve incluso en un bosque
\chapter{Lo esencial es suficiente}

\todo{haiku}

La parte difícil es que lo esencial es también la perfección y, como tal, imposible de alcanzar. Al leer el título de este capítulo se puede pensar que promulgo una vida relajada y sin preocupaciones, por eso quiero aclararlo inmediatamente: nada más difícil que alcanzar lo esencial y, sin embargo, con ello basta para la vida. Las dificultades están ocultas por nuestras propias dispersiones, por nuestra despreocupación por encontrar lo más importante de nuestra vida, nuestra renuencia, incluso miedo, a buscar nuestro camino para, abandonando todo lo demás, seguirlo. En otras palabras no sabemos exactamente qué es lo esencial y siendo así nos cuenta encontrarlo.

% Imagen de un bosque

\lipsum

% --------------------------------------
% La esencia de la vida y el esencialismo
%
% ganar el norte, aún sabiendo cuál es el norte debemos despejar el camino si queremos llegar a él

\backmatter

\appendix

\chapter{Resumen}

\end{document}