\documentclass[12pt,spanish,a4paper,DIV=7,twoside=false]{scrbook}
% fuente base 12, en Español, din A4, la medida del área de texto que prefiero, a una sola cara,

% ------ Línea de 80 caracteres -----------------------------------------------

% Este documento es un Sabia Vida de libro en komascript, preparado para
% trabajar en español

% ------ lista de tareas ------------

\usepackage{todonotes}


% ------ ilustraciones --------------

\usepackage{graphicx}
\DeclareGraphicsRule{.tif}{png}{.png}{`convert #1 `dirname #1`/`basename #1 .tif`.png}


% ------ fuente ---------------------
\usepackage{Alegreya} % U otra fuente que se desee
\addtokomafont{disposition}{\rmfamily} % Encabezados en Serif

\setkomafont{caption}{\footnotesize\itshape} % -- Etiquetes fuente pequeña
\setkomafont{captionlabel}{\usekomafont{caption}}

\usepackage{microtype}  % -- Arreglos Tipográficos

\usepackage[autostyle,spanish=spanish]{csquotes} %-- citas

\usepackage{verse} %-- poemas

% ------ cuadros ---------------------

\usepackage{booktabs}
\usepackage{tabularx}
\usepackage{wrapfig}

% ------ idioma -----------------------

\usepackage[spanish,es-noindentfirst]{babel}  % en Español, por favor
% primer párrafo de cada sección sin sangría
\usepackage[utf8]{inputenc} % codificación de caracteres 
\usepackage[T1]{fontenc}


% ------- referencias -----------------

\usepackage[nottoc]{tocbibind}
\usepackage{cleveref} 
%  Documentación http://tug.ctan.org/macros/latex2e/contrib/cleveref/cleveref.pdf


% ------- enlaces ----------------------

\usepackage[	 colorlinks=true, pdfstartview=FitV, linkcolor=blue, 
                 	citecolor=blue, urlcolor=blue, pdfborder={0 0 0}, 
			   	pdftitle=Sabia Vida, pdfsubject=reflexiones,
				pdfauthor=Miguel de Luis, pdflang=es ES]{hyperref} %enlaces debe ir al final


% ------ meta -------------------------

\title{Sabia Vida}
\subtitle{Reflexiones}
\author{Miguel de Luis}
\date{\today} % o la fecha que sea o \today
\publishers{Pequeña Editorial Invisible}

\begin{document}


% ------ preliminares -----------------

\frontmatter
\maketitle       % --- Inserta página de título
\listoftodos     % --- tareas pendientes 
\todo{borrar lista de todos}
\tableofcontents % --- Inserta índice general

\chapter{Prefacio}

Mi intención es que éste sea el libro más importante de mi vida. No es un intento de sacar dinero de mi blog, ni tampoco recopilar los artículos de más éxito en forma de libro sino una reflexión más profunda y extensa sobre los temas que dieron origen a \textsc{Sabia Vida}

El primer peligro al que me enfrento es caer en algo tan pretencioso como construir una suerte de práxis filosófica para la que carezco de suficientes conocimientos. El segundo, --y al que temo más--, es quedarme corto, conformándome con los lugares comunes y las ideas abreviadas que repiten lo que ya se ha dicho y, en realidad, no necesita volver a decirse. Entre esos dos extremos pretendo situarme, con la esperanza de llegar al final algo de valor que sea digno de transmitirse.

% 140 palabras de 1500

\todo{terminar prefacio}

% ------ contenido principal ----------

\mainmatter

%\lorem[3]

\section{Ligula venenatis}

\lorem[6]

\begin{figure}
\begin{enumerate}
\item Aliquam convallis fringilla libero
\item et vestibulum eros viverra vitae
\item Donec semper hendrerit sapien in tempus
\item sit amet placerat nulla condimentum
\item Donec facilisis tempor arcu
\item Sed consequat finibus gravida
\end{enumerate}
\caption{Lista Numerada}
\end{figure}

\lorem[3]

\chapter{Duis mi velit}

\section{Cras cursus ante tellus}
\lorem[3]

\section{Nam vel finibus}

\lorem[8]


\chapter{¿Para qué vivir?}

Suena a pregunta de tristeza, pero que debería responderse, o al menos plantearse en tiempos de serenidad, y también de alegría, --sobre todo, de alegría. Sin embargo, casi nunca profundizamos en la pregunta, porque la respuesta se piensa, o mejor dicho, se \emph{siente} obvia en la alegría y la tristeza, e insondable en la serenidad. Mientras tanto, la vida sigue fluyendo, con, sin o contra nuestra consciencia. Es como si estuviéramos en una fiesta sin saber para qué estamos, como sin sufriéramos sin saber dónde está el dolor, o cómo si saltáramos de pensamiento en pensamiento.

Estoy convencido de que, --salvo en términos muy genéricos--, debo abstenerme de \emph{dar-te la} respuesta a esta cuestión. En otras palabras no voy a intentar pensar por ti, porque no aspiro a imposibles. Me conformaré con compartir mis meditaciones.
\chapter{Lo esencial es suficiente}



\backmatter

\appendix

\chapter{Resumen}

\end{document}