\chapter{¿Para qué vivir?}

\begin{figure}
\settowidth{\versewidth}{sus sombras las persiguen }
\begin{verse}[\versewidth]
\vin las nubes pasan \\
sus sombras las persiguen \\
\vin el cielo queda
\end{verse}
\end{figure}

Suena a pregunta de tristeza, pero que debería responderse, o al menos plantearse en tiempos de serenidad, y también de alegría, --sobre todo, de alegría. Sin embargo, casi nunca profundizamos en la pregunta, porque la respuesta se piensa, o mejor dicho, se \emph{siente} obvia en la alegría y la tristeza, e insondable en la serenidad. Mientras tanto, la vida sigue fluyendo, con, sin o contra nuestra consciencia. Es como si estuviéramos en una fiesta sin saber para qué estamos, como sin sufriéramos sin saber dónde está el dolor, o cómo si saltáramos de pensamiento en pensamiento.

Estoy convencido de que, --salvo en términos muy genéricos--, debo abstenerme de \emph{dar-te la} respuesta a esta cuestión. En otras palabras no voy a intentar pensar por ti, porque no aspiro ni a imposibles ni a estupideces. Me conformaré con compartir mis meditaciones, mostrarte a dónde yo he llegado, lo que he meditado, no para imponer mi visión y mi reflexión, pues no es perfecta ni para mí mismo sino para animarte a caminar también en este viaje.

\section{La fuente de la vida}

\begin{figure}
\blockquote[Sheldon Cooper]{
Trilema de Münchhaüsen: o la razón depende de una serie de subrazones, lo que lleva a un regresión infinita o se basa en determinados axiomas arbitrarios o forma un ciclo en sí misma, por ejemplo, me quiero ir porque me quiero ir.}
\end{figure}

La explicación del insigne y ficticio Dr Sheldon Cooper sobre el trilema de Münchhaüsen me sirve para facilitar la idea de que carecemos de un método científico para determinar los axiomas de la ciencia. En otras palabras debemos aceptar una base que nos parezca coherente, empezar a construir sobre ella y, si todo falla, volver a empezar. Si esto fuera la entrada de un \emph{blog} lo dejaría aquí, o me bastaría insistir un poco en la idea, añadiendo alguna vía para explorar esa fuente, quizás con la añadidura de la palabra Dios, pero aquí voy a ser mucho más osado. Lo habréis visto venir, la fuente de la vida, o, si queréis, la fuente de la vida que he encontrado, es Dios.

Dios y, en particular, el Dios cristiano, no es tendencia en occidente, pero debo mantenerme firme. Si fuera budista hablaría de el Buda y de sus enseñanzas y sería tu elección aceptarlas o rechazarlas; en todo o en parte. Pero soy cristiano y como tal debo hablar, sin pedir perdón por ello.

Encontré mi fuente no sin esfuerzo; no sé si tú pretendes que la tuya te venga caída del cielo, mientras estás dormido o distraído en otras cosas.

\section{La brújula}

Una vez que encuentres la fuente tendrás, como regalo, una brújula

% Una brújula sirve incluso en un bosque