\chapter{Tú y yo, nosotros y los otros}

\todo{Haiku de tú, yo, nosotros y los otros}

Si acudimos a la cultura popular parece que, --más que en sociedad--, vivimos como una serie de individuos que, da la casualidad, viven cerca los unos de los otros pero sin verdadera interacción. Vayamos a un blog de \emph{superación personal}, lo que encontramos son consejos para mejorarse a uno mismo, y, como mucho, a su propia familia. Los libros de auto-ayuda, de organización personal siguen esa misma línea: cómo mejorarte a ti mismo, sin ocuparse apenas de cómo mejorar el mundo, o al menos tus alrededores y dedicando poco espacio a cómo integrarse con otras personas en una comunidad.  Los mecanismos de \enquote{crowdfunding} agrupan a personas que coinciden en un interés concreto pero que se limitan a entregar dinero. Incluso lo que pasa por comunidades \emph{online} consisten en unos pocos que aportan cosas, copiándolas de otras partes, y unos muchos que simplemente consumen esos contenidos de forma más o menos relajada. Al mismo tiempo las formas de comunidad más tradicionales si bien no han desaparecido, ni muchísimo menos, han ido perdido fuerza, influencia y adhesión. La razón de este capítulo es compartir mi reflexión sobre este asunto para, si no resolverlo, al menos ser capaz de señalar el problema.

\section{Tú}

Tú es siempre misterioso para el yo. El ego, que cree tener la vocación de poder controlarlo todo, aprehenderlo todo y saberlo todo, pero que, en realidad, apenas se controla a sí mismo, tropieza en el tú con un muro infranqueable. Frustrado, por no poder conocer perfectamente a la otra persona, la reduce a unas cuantas ideas, etiquetas si quieres, y tacha de raro a la persona cuando las etiquetas que le ha asignado difieren de las suyas.

\section{Tú y yo}

\section{Nosotros}

\section{Los otros}

\section{Algos y algas}

\section{Todos, ¿juntos?}

% Tratamos de superar aquí la obsesión contemporánea con mejorarme yo a mí mismo. Hablamos también de la relación con otras personas, incluídas lejanas y <<enemigos>>

\section{Lorem, lorem, lorem}

\lipsum