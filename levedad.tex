\chapter{Alegre defensa de la levedad}

\begin{figure}
\begin{description}
\item{Ligereza}\\
De ligero.\\
1. f. Presteza, agilidad.\\
2. f. Levedad o poco peso de algo.\\
3. f. Inconstancia, volubilidad, inestabilidad.\\
4. f. Hecho o dicho de alguna importancia, pero irreflexivo o poco meditado.\\
\end{description}
\end{figure}

La definición de \emph{levedad} que tanto la Real Academia de la Lengua, como el Diccionario de uso del Español se remiten a la de \emph{ligereza} y ésta última parece tener mal prensa. La ligereza, y con ella la levedad, se nos presentan como cosa de tontos, de niños, o de apresurados. Parece ser precisamente lo contrario de lo defiendo tanto en mi blog como en mi libro. Es más, a veces pasa, que uno se lleva una idea equivocada de las palabras, a lo mejor se queda uno con el uso creativo que le ha dado un escritor, o escuchado en una película y acaba por adoptar el significado erróneo por el real. No estoy totalmente seguro de que eso no me haya pasado tmabién en este caso, pero aún a riesgo de empecinarme, ni voy a buscar otro término, ni inventarme otro, ni proponer otro significado. Deja, en vez de eso, que me explique.

Vivimos en una gigantesca nave espacial, un mundo que, a velocidades increíbles viaja, con todo el resto de la galaxia, en rumbo de encuentro, --que no de colisión, no en el sentido usual--, con Andrómeda. Al mismo tiempo damos vueltas junto con todo el resto del sistema en torno al centro de la Vía Lácteca, --con toda probabilidad un agujero negro--, y, casi por último, en torno al Sol, una estrella que, pasados unos miles de millones de años, se hinchará hasta \enquote{morir}. Las otras estrellas, --a las que para entonces podremos haber emigrado-- acabarán, tras procesos distintos, un destino en el que dejarán de producir luz y calor. 

El mismo universo es leve. No solo por lo inconstante de su futuro sino por que, en densidad, es efectivamente leve. Casi todo el universo es espacio ocupado, en todo caso, por espuma cuántica, algún átomo de hidrógeno despistado o, para la mayoría de los efectos, vacío. Además la materia que conocemos, con la honrosa excepción de las estrellas de neutrones, consiste en poco más que vacío ligeramente contaminado de partículas.

Sin irnos tan lejos en el tiempo, ni tan alejados de nosotros mismos, ¿qué hay más leve que el propio ser humano? 

\section{La alegre humildad}

La levedad que propongo es alegre humildad, y consiste en la aceptación de la verdad de la condición humana, frágil, pecadora, siniestra a veces, siempre inconstante, con aspiraciones inmensamente mayores que sus capacidades y, sin embargo, capaz de sorprendernos con las más grandes hazañas cuando solo cabía esperar mediocridades. Y esto es predicable tanto de la humanidad en su conjunto, como de cada especimen de nuestra especie. Quizás, la condición humana no sea otra cosa, que el reflejo de la condición del universo.

Ante esta verdad hay tres opciones: la soberbia, que niega los límites o se exige superarlos, a pesar de ser imposibles, la humildad resignada, que acepta los límites pero evitar pensar en ellos o sentirlos con cualquier sentimiento que no sea el de tristeza, y, la alegre humilidad.

% Somos cosa leve, cosa pequeña, cosa humilde, como los niños. ¡Juguemos!